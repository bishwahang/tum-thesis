\documentclass[]{beamer}

% for themes, etc.
\mode<presentation>
{ \usetheme{boxes} }
\usepackage{booktabs}
\usepackage{times}  % fonts are up to you
\usepackage{graphicx}
\usepackage[utf8]{inputenc}
\usepackage{listings}
\usepackage{hyperref}
\usepackage{algorithmic,algorithm2e,float}
\usepackage{url}

\usepackage{caption}
\lstset{
basicstyle=\small\ttfamily,
columns=flexible,
breaklines=true
}
\graphicspath{{./figures/}}
\usepackage{amsmath}
\usepackage{amsthm}
\usepackage{mathtools}
\setbeamercolor{footline}{fg=blue}
\setbeamerfont{footline}{series=\bfseries}
\addtobeamertemplate{navigation symbols}{}{%
    \usebeamerfont{footline}%
    \usebeamercolor[fg]{footline}%
    \hspace{1em}%
    \insertframenumber/\inserttotalframenumber\
}
\usepackage{beamerthemeshadow}

\newcommand{\HRule}{\rule{\linewidth}{0.5mm}}
% these will be used later in the title page
\title[``Botnet Battlefield'']{``Botnet Battlefield'': A Structured Study of Behavioral Interference Between Different Malware Families}
\author[Bishwa Hang Rai]{
Bishwa Hang Rai
\\{\small Supervisor: Prof.\ Dr.\ Alexander Pretschner}
\\{\small Advisor: Mr.\ Tobias Wüchner}}
%\author{Bishwa Hang Rai}

\institute{\includegraphics[width=0.15\textwidth]{logos/tum.png}~\\[1cm]Department of Informatics\\
TU München}
\date{January 22, 2016}
% \setbeamertemplate{caption}[numbered]
% note: do NOT include a \maketitle line; also note that this title
% material goes BEFORE the \begin{document}

% have this if you'd like a recurring outline
\AtBeginSection[]  % "Beamer, do the following at the start of every section"
{
\begin{frame}<beamer>
\frametitle{Outline} % make a frame titled "Outline"
\tableofcontents[currentsection]  % show TOC and highlight current section
\end{frame}
}
\begin{document}

% this prints title, author etc. info from above
\begin{frame}
\begin{titlepage}
\end{titlepage}
\end{frame}
\frame{\frametitle{Table of contents}\tableofcontents}
\section{Introduction}
\subsection{Background}
\label{sub:Background}
\begin{frame}[t]{Malware}
\begin{itemize}
    \item Malware is a general term to refer any malicious software that corrupts or steals data, or disrupt operations with illegitimate access to computer or computer networks
    \item It can be classified into self replicating and non-replicating
    \item Self replicating can make copies of themselves e.g. \emph{Virus and Worms}
    \item Non-replicating cannot make copies of themselves e.g. \emph{Trojans}
\end{itemize}
\end{frame}
\begin{frame}[t]{Malware}
  \begin{itemize}
    \item Based on its ability to change its structure it can also be broadly classified into Polymorphic and Metamorphic
    \item Polymorphic uses encryption and code obfuscation (dead-code insertion, subroutine reordering, instruction substitution) techniques
    \item Metamorphic malware uses only code obfuscation
    \item Different variants of same malware with same semantics or from same author are regarded as to be from same family.
  \end{itemize}
\end{frame}
\begin{frame}[t]{Growth of Malware}
  \begin{itemize}
    \item With the increase in growth of the Internet, many of our daily life activities such as email, banking, bill payment, and social networking are dependent on it.
    \item Malware authors are introducing new malware on daily basis to steal those valuable data and personal information and sell it illegally in the underground market.
    \item Annual loss caused by malware in 2006, 2.8 billion dollars in US and 9.3 billion euros in Europe
    \item Driven by monetary profit, high rise in numbers of new malware with 140 million new malware introduced in 2015 alone
  \end{itemize}
\end{frame}
\begin{frame}[t]{Interference Between Malware Families}
  \begin{itemize}
    \item There has been some anecdotal evidences of feud between the malware families
    \item In 2004, NetSky vs Bagle and MyDoom trying to remove each other along with message of profanity
    \item In 2010, SpyEye vs Zbot with KillZeus feature
    \item In 2015, Shifu malware family with AV like feature
    \item All of these interferences were to negate the presence of another malware
    \item Increase their own profit taking control of larger share of economy
  \end{itemize}
\end{frame}
\subsection{Problem Statement}
\label{sub:Problem Statement}
\begin{frame}[t]{Problem Statement}
\begin{itemize}
  \item The purpose of our research is to identify the existence of aforementioned behavioral interference between the malware families
  \item The study will provide novel knowledge for understanding the dynamic aspect of modern malware, the inter-family relations, and their associated underground economy
  \item This behavior is also a case for environment-sensitive malware
  \item That is to say malware changing their behavior depending on different factors of their running environment, such as presence or absence of files, programs, or running services
\end{itemize}
\end{frame}
\begin{frame}[t]{Research Process}
\begin{itemize}
\item Get wide variety of malware samples
\item Use heuristics and clustering to get the candidate pair list
\item Run each candidate pair in malware analysis system (Anubis in our case)
\item Analyze the log of analysis run to detect behavioral interference
\end{itemize}
\begin{figure}[h]
    \centering
    \def\svgwidth{\columnwidth}
    \scalebox{0.5}{\input{figures/overview.pdf_tex}}
\end{figure}
\end{frame}

\subsection{Contribution}
\label{sub:Contribution}
\begin{frame}[t]{Contribution}
Our research will provide the following contributions:
\begin{itemize}
  \item To the best of our knowledge, we are the first to perform a systematic study of interferences between malware families\\
    % Our work is based on wide variety of a large malware dataset collected over time in the wild.\\
    % This novel research will help better understand the dynamic aspect of malware behavior.
  \item A novel approach to malware clustering based on malware behavior profiles
  \item An automated system that detects interfering malware samples on a large scale
\end{itemize}
\end{frame}
\section{Methodology}
\label{sec:Methodology}
\begin{frame}[t]{title}
  
\end{frame}

\section{Evaluation}
\label{sec:Evaluation}
\subsection{Experiment}
\label{sub:Experiment}
\begin{frame}[t]{List of Candidate Pairs}
\begin{columns}
\begin{column}{4.7cm}
\begin{itemize}
  \item Value of N (maximum family cutoff) in algorithm chosen to be 10
  \item File with the highest number of candidate pair and Process the lowest
  \item No candidate pair from resource type Job, Device, Driver
\end{itemize}
\end{column}
\begin{column}{5.3cm}
  \begin{tabular}{l l l l}
    \toprule
    Resource types & \#candidate pairs\\
    \midrule
    File & 213,171 \\
    Registry & 39,899 \\
    Sync & 7,781 \\
    Section & 2,786 \\
    Process & 54\\
    \bottomrule
    Total & 263,691\\
  \end{tabular}
\end{column}
\end{columns}
\end{frame}
\begin{frame}[t]{Experiment Setup}
  \begin{itemize}
    \item 7 Anubis instance
    \item Each instance emulates entire running PC with Windows XP Service Pack 3 as OS
    \item Uses Qemu and monitors process by invoking callback routine for every basic block executed in virtual processor
    \item Unpacker and Packer used to run the candidate pair
    \item 10 minutes as total run time of each candidate pair experiment
    \item 4 minute for each malware, and 2 minute to boot system
  \end{itemize}
\end{frame}
\begin{frame}[t]{Result of Candidate Run}
  \resizebox{\columnwidth}{!}{%
  \begin{tabular}{l l l l}
    \toprule
    Resource types & \# tested pairs & \# true positive & prediction accuracy\\
    \midrule
    File & 5,000 & 1032& 20.64\%\\
    Registry & 5,000 & 731& 14.62\%\\
    Sync & 1,000 & 119& 11.9\%\\
    Section & 1,000 & 93& 9.3\%\\
    Process & 54 & 6& 11.11\%\\
    \bottomrule
    % Total & 263691\\
  \end{tabular}%
}
\begin{itemize}
  \item Highest Accuracy for File and Registry
  \item Lowest for Process
  \item Average accuracy rate $14.25\%$
\end{itemize}
\end{frame}
\begin{frame}{Some Examples}
  \begin{itemize}
    \item Artemis! vs Cosmu on resource \url{C:\\Old.exe}
    \item VB.CB vs Startpage.AI on resource \url{C:\\WINDOWS\\window.exe}
    \item KeyLogger vs OnlineGames on resource \url{C:\\windows\\system32\\svrchost.exe}
  \end{itemize}
  
\end{frame}
\subsection{Threats to Validity}
\label{sub:Threats to Validity}
\begin{frame}{Threats to Validity}
  \begin{itemize}
    \item Different values of N would give different candidate pairs and different results
    \item Random resource name
    \item Total execution time 10 minutes
    \item Sequence of execution
    \item Semantics of Malware
  \end{itemize}
\end{frame}


\section{Conclusion}
\subsection{Summary}
\label{sub:Summary}
\begin{frame}{Summary}
  \begin{itemize}
    \item Behavioral interference between malware families exists
    \item Malware checks for the presence of resource created by other malware and deletes it
    \item Our system could detect such interfering malware with average accuracy rate of $14.25\%$
    \item In our dataset, Files and Registries were the most interfered resource and Process was the least
  \end{itemize}
\end{frame}
\subsection{Future Work}
\label{sub:Future Work}
\begin{frame}{Future Work}
  \begin{itemize}
    \item Make the experiment more efficient to run multiple times with different parameters
    \item Research on other different approaches to clustering
    \item In depth analysis (static) of positive pair to know the true semantics of malware
  \end{itemize}
\end{frame}
\end{document}
