\chapter{Introduction}\label{chapter:introduction}
Malware, short for malicious software, is a generic term referring to any illegitimate software that can cause damage or steal the data, disrupt the operation, or gain unauthorized access, on a computer or computer network~\cite[]{ciscodif}.
Malware can infect the system by being bundled with some other program or being attached as macros in a file.
When a user opens such a program or file, the malware gets installed onto the user's machine.
Malware can also install themselves by exploiting some common vulnerabilities in an operating system (OS), a network device, or other common software.
A vast majority of malware is installed by some action by the user, such as clicking an e-mail attachment or downloading a file from the Internet~\cite[]{ciscodif}.\\

Malware can be classified into self-replicating, such as \emph{viruses and worms}, and non-replicating, such as \emph{Trojans}~\cite[]{malsoft}.
Unlike a virus, which needs user action (opening the infected file), worms are capable of automated replication and propagation, through computer networks~\cite[]{malsoft}.
Trojans disguise themselves to be legitimate and harmless, but have hidden malicious functions.
A malware can also be categorized into \emph{polymorphic} and \emph{metamorphic}, by its ability to change its code-structure every time it infects a new victim.
\emph{Polymorphic} malware uses code obfuscation (e.g.\ dead-code insertion, subroutine reordering, instruction substitution) and encryption to create a new variant of itself, whereas,
\emph{metamorphic} malware uses only code obfuscation to change its code-structure~\cite[]{rad2011evolution,rad2012camouflage}.
However, the semantics of the program (malware) remains same.
We discuss in details about types of malware in~\autoref{sec:Malware Types and Families}.\\

As the Internet increased growth over the last two decades, more and more devices are now connected to the Internet.
Many of our daily life activities are now depended on its usage, such as email, banking, bill payment, and social networking.
Along with this, the underground Internet economy, illegal trading of valuable data and personal information, has also been on the rise.
Malware authors now do not write malware just for fun, to create annoyance or break into some system for bragging rights, but also for profit.
Malware authors look for banking credentials, credit cards, and personal information that they can gather and sell in the underground market.
In 2006, annual losses caused by malware were estimated to be 2.8 billion dollars in United States and 9.3 billion euros in Europe~\cite[]{moore2009economics}.
Driven by the high profit and rise in easily available tools to create polymorphic and metamorphic malware, there has been rapid increase in the number of new malware being introduced each day~\cite[]{tian}.
According to \emph{AV-Test}, an independent IT-Security Institute based in Germany, about 140 million new malware were introduced in 2015 alone, making the total number of malware recorded so far almost half a billion~\cite[]{avtest}.\\

Malware has become a profitable business.
Malware authors are now vying with each other to take sole control of their victim for larger profit.
Many cases have been found, where one malware tries to delete another malware from the victim's machine.
Further, once they have infected the victim, they prevent the host from being infected by other malware.
In 2010, \emph{SpyEye}, a Trojan, was found to have a feature \emph{KillZeus} that would remove \emph{Zeus}, which is also another Trojan, making \emph{SpyEye} the only malware to run in the compromised system~\cite[]{sanszeus}.
In 2015, \emph{Shifu}, a banking Trojan, was found to have a similar behavior as Anti Virus (AV), i.e., preventing the host from further infection by other malware, once it had taken control over the victim's machine~\cite[]{secintelshifu}.
We have discussed in details, about such anecdotal evidence of interaction between malware families, in \autoref{sec:Interference between Malware Families}.
\section{Problem Statement}
\label{sec:Problem Statement}
The purpose of our research is to identify the existence of aforementioned behavioral interference between the malware families.
The study of behavioral interference between malware families will provide novel knowledge for understanding the dynamic aspect of modern malware, the inter-family relations, and their associated underground economy.
This behavior is also a case for environment-sensitive malware, where malware change their behavior depending on different factors of their running environment, such as presence or absence of files, programs, or running services.
To the best of our knowledge, there hasn't been a study of such interference between two malware families so far.
We present a systematic way to find out such interferences from a large number of malware samples.
We analyze our malware sample dataset to find the malware pairs that could possibly exhibit the interfering behavior with each other.
The malware pair is then run in malware analysis system together and examined to detect any behavioral interference.
% In the following section, we describe how we plan to implement this approach.
\section{Research Process}
\label{sec:Research Process}

\begin{figure}[h]
    \centering
    \def\svgwidth{\columnwidth}
    \scalebox{0.7}{\input{figures/overview.pdf_tex}}
    % \input{figures/overview.pdf_tex}
\caption{System Overview}
\label{fig:overview}
\end{figure}

A holistic overview of our research approach is shown in~\autoref{fig:overview}.
To perform the research, we worked with dataset of {\gettotalmalwarei{}} malware samples, collected overtime (2007--02--07 to 2012--09--10) by \emph{Anubis}~\cite[]{anubis}, a dynamic full-system-emulation-based malware analysis environment, from its public web interface and a number of feeds from security organizations.
Detailed discussion on \emph{Anubis} and malware analysis is done in~\autoref{sec:Malware Analysis}.\\

Randomly picking a malware pair to analyze from the available dataset would not be a good approach, whereas analyzing all the possible pairs of millions of malware is not scalable.
We use heuristics and clustering techniques to filter our dataset and get list of malware pairs that has high probability of exhibiting the behavioral interference.
We call those malware pairs \textbf{candidate pairs}.
The candidate pairs, which is the subset of whole dataset (thousands in numbers), are then run in malware analysis system, and the results are further analyzed to detect if there is any behavioral interference between the malware samples.\\

We generate a list of probable candidate pairs, from our malware dataset, based on the common resource that the malware sample operates (create, access or delete) on.
\textbf{Resource} refers to any entity, such as Files, Registry, Section, Sync, that can be modified or queried by malware using system calls.
Since \emph{Anubis} emulates the Windows OS, \textbf{system calls} refers to the Windows Application Programming Interface (API) functions.
The trace of all the resources modified or queried by malware during its execution is referred to as the \textbf{behavioral profile} of that malware.
In \autoref{sub:Behavioral Profile} and \autoref{sub:Resource Types and Activities} we define behavioral profiles and different resource types in detail respectively.\\

In order to minimize our search space for candidate pair selection, and maximize the probability for finding a pair with behavioral interference, we take into account only those malware which would create a resource successfully and another malware that tries to access or delete that same resource, but with a failure.
If malware makes a failed attempt to access or delete resources created by another malware, there should be some interaction between those malware, as these actions are suspicious.
It is trying to detect the presence of other files or registries in the environment it is currently running, and its behavior is affected by the positive or negative detection.
This shows the dynamic aspect of malware behavior and its environment-sensitive nature.\\

The simple heuristic of finding interfering malware candidate pairs based on common resource gave us concrete candidate pairs.  The candidate pairs were still too large to run in a scalable manner and find substantial results.
In order to further study the dataset, we used clustering of malware (based on behavior of malware).
\textbf{Malware clustering} is the process of grouping a set of malware in such a way that malware in the same group are more similar, based on the behavioral profile in our case, to each other than to those in other group.
Such groups are called \textbf{clusters}.
Clustering of malware based on behavioral profiles has been used extensively in previous research, which we have listed in details in \autoref{sec:Malware Clustering}.
We treat each cluster as a family, and malware falling in those clusters as variants of that family.
Candidate pairs are chosen in such a way that two malware representing a pair, do not belong to the same family.
The algorithm for candidate selection based on clustering results and the common resource heuristic is described in~\autoref{sub:Candidate Selection}.
Each candidate pair from the list of candidate pairs were then run in the \emph{Anubis} system for analysis.\\
Two malware sample belonging to a single candidate pair were run in the \emph{Anubis} system consecutively, with a time interval between the two run.
The procedure has been described in detail in~\autoref{sub:Running the Candidate}.
The result of Anubis was then analyzed to see if one malware deletes the resource created by another, which provides proof for behavioral interference between the two malware.
The details of result analysis is described in~\autoref{sub:Result Analysis}.\\
% An \emph{unpacker} was designed in order to run a candidate pair inside the \emph{Anubis} system for analysis.
% The \textbf{Unpacker} drops the candidate pair into the analysis system and executes both malware with a set time interval in between two executions [see~\autoref{sec:packerunpacker}].
% We were able to successfully find X pairs of malware with interfering behavior with a true positive rate of X.\\
\section{Contribution and Structure}
\label{sec:contributionstructure}
Our research will provide the following contributions:
\begin{itemize}
  \item To the best of our knowledge, we are the first to perform a systematic study of interferences between malware families.
    Our work is based on wide variety of a large malware dataset collected over time in the wild.
    This novel research will help better understand the dynamic aspect of malware behavior.
  \item A novel approach to malware clustering based on malware behavior profiles.
    % We show our clustering approach is good at grouping similar behaving malware in to the same cluster.
  \item An automated system that detects interfering malware samples on a large scale.
  % Our system was able to find malware pairs with interfering behavior with good accuracy.
  % TODO: number of pairs
\end{itemize}
We discuss malware types and families, evidence of interaction between malware families, malware analysis system, previous works on the use of malware clustering based on malware behavior generated from dynamic malware analysis, and motivation for our research in \autoref{chapter:literature_review}.
In \autoref{chapter:methodology}, following the motivation, we give an overview, design rationale, and give a technical description of our work flow and system.
In \autoref{chapter:implementation}, we explain how we implemented the system, problems faced during the implementation, and how we solved them.
We present the findings and evaluation of our work in \autoref{chapter:evaluation}.
Finally, we conclude with a discussion of our work and possible improvement in future in \autoref{chapter:conclusion_and_future_work}.
