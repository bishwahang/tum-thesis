%!TEX root = ../main.tex
\chapter{Evaluation}
\label{chapter:evaluation}
So far we implemented the system to find the minimal set of candidate pairs from our test dataset that has probability of exhibiting behavioral interference.
In this section, we describe how we test those candidate pairs in the Anubis system to detect if the behavioral interference exists between the malware samples and evaluate the effectiveness of our approach.
% In this section we describe how we experiment those candidate pair and evaluate the results.
\section{Candidate Pairs}
\label{sec:Candidate Pairs}
With our approach, in total we found \emph{263,691} candidate pairs with probable behavioral interference.
The breakdown of the number is shown in~\autoref{tab:totalcandidates}.
We set the maximum number of families, the value of \emph{`N'} in Algorithm~\autoref{alg:Candidate Selection}, to be 10.
Any interesting resource, with corresponding malware families count for create or access/delete more than 10, were discarded.
We do this because any resource associated with too many families is less interesting [see~\autoref{sub:Candidate Selection}].

\begin{table}[ht]
  \caption[Number of probable candidate pairs w.r.t. Resource types]{Number of candidate pairs w.r.t. Resource types}\label{tab:totalcandidates}
  \centering
  \begin{tabular}{l l l l}
    \toprule
      % \bf{Resource Types} & \bf{\# of candidate pairs}  & \bf{true positive} \\
    Resource types & \# candidate pairs\\
    \midrule
    File & 213,171 \\
    Registry & 39,899 \\
    Sync & 7,781 \\
    Section & 2,786 \\
    Process & 54\\
    \bottomrule
    Total & 263,691\\
  \end{tabular}
\end{table}
The largest number of candidate pairs was \emph{213,171} for resource type \emph{File} and the lowest number was \emph{54} for for resource type \emph{Process}.
Registry resource type was second with number of candidate pairs \emph{39,899}.
Sync and Section resource type were third and fourth with number of candidate pairs \emph{7,781} and \emph{2,786} respectively.
With our data, we can say that file and registry are the most modified resource type, whose presence in the system is checked by another malware.\\


Although, in our dataset, we had taken resource types {\getresourcetypes{}} into consideration, our algorithm did not find any candidate pair with behavioral interference based on resource type \emph{Job, Device, and Driver}.
This infers that in our dataset, there were no two malware sample from two different families, where one malware created a Job, Device or Driver and another malware accessed or deleted it.
\section{Running the Experiment}
\label{sec:Running the Experiment}
We had remote access to 7 instances of Anubis system for running candidate pairs.
Candidate pair were packed together using the \emph{`Pakcer'} and dropped them inside Anubis system using the \emph{`Unpacker'}as described in~\ref{sec:packerunpacker}.
The time delay between the execution of two binaries in the \emph{Unpacker} was set to 4 minutes.
This allowed the first binary enough time to perform its resource activities before the second binary starts.
The total time set for a single Anubis run was set to 10 minutes.
Some of the missing binary from the candidate pairs were downloaded from \emph{VirusTotal}.\\

We sampled malware pairs randomly from the probable candidate pairs list and run them in Anubis system to test for behavioral interference.
The number of sampled candidate pair and the result of the experiment is shown in~\autoref{tab:resultscandidate}.
% The time of execution for each worker was set to 10 minutes, when we ran the experiment such that malware `a' of candidate pair $(a,b)$, run first for 3 minutes, and after that malware `b' of the same pair is run in the same anubis worker.
% The time of execution were changed accordingly with respect to experiment.
% When we ran the malware `a' first, then wait for some minutes for malware `b' to run, and again wait for some minutes before malware `a' was run again, the anubis worker run time was set to 9 minutes.
% The result of the Anubis run were then copied to local machine to analyze further.
\begin{table}[ht]
  \caption[Results of candidate pair run]{Results of candidate pair run}\label{tab:resultscandidate}
  \centering
  \begin{tabular}{l l l l}
    \toprule
    Resource types & \# tested pairs & \# true positive & precision\\
    \midrule
    File & 2000\\
    Registry & 1000\\
    Sync & 1000\\
    Section & 1000\\
    Process & 54\\
    \bottomrule
    % Total & 263691\\
  \end{tabular}
\end{table}
\section{Effectiveness}
\label{sec:Effectiveness}
\section{Efficiency}
\label{sec:Efficiency}


\section{Threats to Validity}
\label{sec:Threats to Validity}
The results we presented is in the context of our approach with dependent variables which we controlled throughout the experiment.
In this section we discuss such threats to validity of our results.

% Firstly, we only took into consideration eight resource types ({\getresourcetypes{}).
% We chose this based on k
During candidate selection, we chose the cutoff number for families to 10.
Different values of N would drastically change the candidate pairs list generated from the candidate selection algorithm [see Algorithm~\autoref{alg:Candidate Selection}].
This can change the statistics of results.
We also did not tackle the fact that many resources can have randomly generated names.
The resource name we considered for the selection of malware pair, might not be generated next time we run the malware pair for testing.
We could have increased the quality of probable candidate pair selection if we had filter out malware pair with such corresponding resource.
% Hence our list of candidate pairs is dependent on the choice of number of N, and result values could be 
% However, this did not effect the quality of the candidate pair selected, but we could have filtered out many fals

When we ran the experiment on those candidate pairs, the number of delay between the two executions were set to be 4 minutes.
We chose the delay times after running some initial candidate pairs and recording their time of execution.
Some malware might execute only for few seconds and some executed for few minutes.
We chose 4 minutes based on the highest time taken by our test malware samples to finish its execution.
The reason was to let the first malware run completely and create all the resources it is supposed to.
But, there might be some malware samples that runs more than 4 minutes.
We did not know the exact maximum time, and increasing the time delay arbitrarily would mean increasing the time of experiment for each candidate pair without certainty to get better results, which we avoided.

We ran the candidate pairs just once and did not repeat the experiment to see if we get the similar results.
We did run few malware pair with positive result more than once, and the result was same (deletion of resource occurred).
Also, we only ran the candidate pair, \emph{(a,b)}, in single order of execution, such that \emph{`a'} runs first, followed by \emph{`b'}, where \emph{`a'} was supposed to create the resource and \emph{`b'} was supposed to access/delete it.
We do no know if the behavioral interference persists if we run the candidate pair in reverse order; with \emph{`b'} being executed first followed by \emph{`a'}.
We ran few candidate pairs with positive result as test and found that the deletion of resource did not occur if the order of execution was reversed.
However, for the both of the above cases, we cannot confirm that our results will always be same, as we did not perform the test in all malware pairs with positive result of behavioral interference; because of the time bound.

We found the concrete evidence on deletion of resource, but we still do not know the true semantics of the malware.
The malware could have deleted just one single resource that could be very common such as \emph{`Desktop.ini`} file.
Further analysis of individual malware belonging to the pair must be done and correlated in order to confirm that the purpose of malware was indeed to negate the infection of other malware from the system.
\section{Summary}
\label{sec:Summary}

We have shown with our results that our system can find malware pair from different families with behavioral interference.
During the experiment, there were  many aspects that we overlooked and many variables that we tuned to get the current results.
But, those threats only effect the effectiveness and efficiency of our result, and not the fact that there were malware samples which deletes the resources created by another malware.
With this results, we provide strong support to our proof of concept for the existence of such behavioral interference between the malware of different families, and that our system can detect such behavioral interference.

