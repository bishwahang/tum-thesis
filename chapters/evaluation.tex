%!TEX root = ../main.tex
\chapter{Evaluation}
\label{chapter:evaluation}
Our hypothesis is the existence of behavioral interference between malware samples of different families.
So far we implemented the system to find the minimal set of candidate pairs, from our test dataset, that has probability of exhibiting behavioral interference.
Now we generate the candidate pairs and execute them together in the Anubis system for analysis.
We believe that when malware sample of the candidate pair are run together, the first malware supposed to create the resource corresponding to the candidate pair will do so, and the second malware will delete that specific resource created by the former malware.

In this chapter, we first give a detail break down of candidate pairs that we generate from our algorithm in~\autoref{sec:Candidate Pairs}.
Then, we describe the specifications of Anubis system and experiment setup in~\autoref{sec:Experiment Setup}.
We show our results and discuss the effectiveness of our approach and some interesting cases in~\autoref{sec:Results}
We elaborate on the threats to validity of our approach in~\autoref{sec:Threats to Validity}.
Lastly, we end this chapter with summary in~\autoref{sec:Summary}.
% We end the chapter with threats to validity and summary of the results.

\section{Candidate Pairs}
\label{sec:Candidate Pairs}
To find the candidate pairs, the maximum number of families, the value of \emph{`N'} in Algorithm~\autoref{alg:Candidate Selection}, was set to 10;
i.e., any interesting resources with more than 10 corresponding malware families, associated by create or access/delete operation, were discarded.
We do this because resource associated with too many families is less interesting [see~\autoref{sub:Candidate Selection}].
With our approach, in total we found \emph{263,691} candidate pairs with probable behavioral interference.
The breakdown of the number is shown in~\autoref{tab:totalcandidates}.

\begin{table}[ht]
  \caption[Number of probable candidate pairs w.r.t. Resource types]{Number of candidate pairs w.r.t. Resource types}\label{tab:totalcandidates}
  \centering
  \begin{tabular}{l l l l}
    \toprule
      % \bf{Resource Types} & \bf{\# of candidate pairs}  & \bf{true positive} \\
    Resource types & \# candidate pairs\\
    \midrule
    File & 213,171 \\
    Registry & 39,899 \\
    Sync & 7,781 \\
    Section & 2,786 \\
    Process & 54\\
    \bottomrule
    Total & 263,691\\
  \end{tabular}
\end{table}
The largest number of candidate pairs was \emph{213,171} for resource type \emph{File} and the lowest number was \emph{54} for for resource type \emph{Process}.
Registry resource type was second with number of candidate pairs \emph{39,899}.
Sync and Section resource type were third and fourth with number of candidate pairs \emph{7,781} and \emph{2,786} respectively.
With our data, we can say that file and registry are the most modified resource type, whose presence in the system is checked by another malware.\\

Although, in our dataset, we had taken resource types {\getresourcetypes{}} into consideration, our algorithm did not find any candidate pair with behavioral interference based on resource type \emph{Job, Device, and Driver}.
This infers that in our dataset, there were no two malware sample from two different families, where one malware created a Job, Device or Driver and another malware accessed or deleted it.
\section{Experiment Setup}
\label{sec:Experiment Setup}
In order to run the candidate pairs, we had remote access to 7 instances of Anubis system.
Anubis uses Qemu, ``a machine emulator which can run an unmodified target operating system (such as Windows or Linux) and all its applications in a virtual machine~\cite[]{qemu}'', as emulator component.
Anubis modifies Qemu's dynamic translation, runtime conversion of target CPU instruction into host instruction set~\cite[]{qemu}, to closely monitor the process under analysis.
Anubis does so by invoking its callback routine before every basic block (sequence of instruction ending in jump statement or changing the CPU state) is executed on the virtual processor~\cite[]{ttanalyze}.
Each of the 7 instance of Anubis system emulates an entire PC computer system, running complete and unmodified version of Windows XP (with Service Pack 2) as an underlying operating system, to provide an accurate environment for malware experiment~\cite[]{ttanalyze}.
\\

Candidate pair were packed together using the \emph{`Packer'} and dropped inside Anubis system for analysis using the \emph{`Unpacker'}, as described in~\ref{sec:packerunpacker}.
Some of the missing binary from the candidate pairs were downloaded from \emph{VirusTotal}.
The time delay between the execution of two binaries in the \emph{Unpacker} was set to 4 minutes.
This allowed the first binary enough time to perform its resource activities before the second binary starts.
The total time for a single Anubis run was set to 10 minutes; 4 minutes for each malware and 2 minutes for PC emulator to boot.
\section{Results}
\label{sec:Results}
We sampled malware pairs randomly from the probable candidate pairs list (\autoref{tab:totalcandidates}) and ran them in Anubis system to test for behavioral interference.
The log of resource activities from the experiment run was then parsed and analyzed as described in~\autoref{sub:Result Analysis} to detect the behavioral interference between the malware samples.
The number of sampled candidate pair and the result of the experiment is shown in~\autoref{tab:resultscandidate}.
\begin{table}[ht]
  \caption[Results of candidate pair run]{Results of candidate pair run}\label{tab:resultscandidate}
  \centering
  \begin{tabular}{l l l l}
    \toprule
    Resource types & \# tested pairs & \# true positive & prediction accuracy\\
    \midrule
    File & 5,000 & 1032& 20.64\%\\
    Registry & 5,000 & 731& 14.62\%\\
    Sync & 1,000 & 119& 11.9\%\\
    Section & 1,000 & 93& 9.3\%\\
    Process & 54 & 6& 11.11\%\\
    \bottomrule
    % Total & 263691\\
  \end{tabular}
\end{table}
\subsection{Effectiveness}
\label{sub:Effectiveness}
With our results shown in~\autoref{tab:resultscandidate}, we can say that our system was able to find interfering malware pair with an average prediction accuracy of $14.25\%$.
Prediction accuracy is the ratio of true positive to tested pairs and average is taken upon all the 5 experimented resource types.
Resource type File had the highest prediction accuracy with $20.64\%$ followed by resource type Registry with $14.62\%$.
We can say that in our dataset more malware pair were showing behavioral interference associated with File and Registry.
But, this also might be true because we had larger sample tested from these two resource type, 5000 each, compared to 1000\ samples in the case of Sync and Section, and only 54 for resource type Process.
\subsection{Interfering Malware Pairs}
\label{sub:Interfering Malware Pairs}
We did a detailed study of the analysis report of few malware pairs with positive result of interfering behavior.
We describe those cases of interference here.
The labeling of the malware with generic family name is done from the VirusTotal labels.
As discussed earlier, there is inconsistency between the labels provided by different AV vendors.
We chose the family name of the malware based on common name given by the most of the AV vendor.
% Another case was with malware pair consisting of malware variants of \emph{Trojan.Win32.Skillis} type.
% The malware variants of \emph{Trojan.Win32.Skillis} family deleted all the files presented in the `\url{C:\\}' folder.
% Thus, deleting everything including the resources created by the malware sample, which it was paired with, and which was run before it.\\
% This is a false positive case which our system did not handle.

% There were normal deletion of resource, created by malware which was run first, by the malware which was run second (after the first malware run), as expected.
We had a case where malware of \emph{Artemis!} family created the file \url{C:\\Old.exe} which was later deleted when malware variants from \emph{Trojan.Win32.Cosmu} was run.
We also had another case for malware pair of variants \emph{Worm:Win32/VB.CB} and \emph{Trojan:Win32/Startpage.AI}.
The \emph{Worm:Win32/VB.CB} malware type created the file \url{c:\\WINDOWS\\window.exe} and also created an autostart in the folder \url{C:\\Documents and Settings\\Administrator\\Start Menu\\Programs\\Startup\\window.exe}.
After the first malware run, when the malware of \emph{Trojan:Win32/Startpage.AI} type was executed, it deleted the file \url{c:\\WINDOWS\\window.exe} created previously by \emph{Worm:Win32/VB.CB} malware type.

There was also an interesting cases where the malware which was supposed to delete the resource, performs the delete operation on that resource, but then later creates a new file with the same name through the rename operation.
This case was between the malware of family \emph{Trojan-Spy.Win32.KeyLogger} and \emph{Trojan-PWS.Win32.OnLineGames}.
The malware of type \emph{Trojan-Spy.Win32.KeyLogger}, which was run first, created the file `\url{C:\\windows\\system32\\svrchost.exe}'. % chktex-file 1
Later, when the malware of \emph{Gen:Trojan.Heur.RP} type was run, it first deleted the current existing \emph{`svrchost.exe'} file created by first malware run, and then renamed the file \emph{`fuc1.tmp'}, which it had created in \emph{Temp} folder, with the new name \emph{`svrchost.exe'}.\\

\section{Threats to Validity}
\label{sec:Threats to Validity}
The results we presented here is in the context of our approach, with dependent variables which we controlled throughout the experiment.
In this section we discuss such threats to validity of our results.

During candidate selection, we chose the cutoff number for families to be 10.
Different values of N would drastically change the list of candidate pairs [\autoref{tab:totalcandidates}] generated from the candidate selection algorithm [see Algorithm~\autoref{alg:Candidate Selection}].
The change in candidate malware sample pairs will also change the statistics of results that we presented above.
We also did not tackle the fact that many resources can have randomly generated names.
The exact resource name we considered for the selection of malware pair might not be generated next time, when we run the malware pair for testing.
% On the contrary, there could be random match of the resource name generated by one malware with resource name trying to access by another malware.
We could have increased the quality of candidate pair selection if we had filter out malware pair with such randomly generated corresponding resource name.
% Hence our list of candidate pairs is dependent on the choice of number of N, and result values could be 
% However, this did not effect the quality of the candidate pair selected, but we could have filtered out many fals

When we ran the experiment on the probable candidate pairs, the number of delay between the execution of two malware was set to be 4 minutes.
We chose the delay times after running some initial candidate pairs and recording their time of execution.
We chose 4 minutes based on the highest time taken by the malware sample from initial test candidate pairs to finish its execution.
The reason was to let the first malware run completely and create all the resources it is supposed to, so that second malware can access it.
But, there can be some malware samples that runs more than 4 minutes.
We do not know the exact maximum malware execution time.
Increasing the time delay arbitrarily would mean increasing the time of experiment for each candidate pair without certainty to get better results, which we avoided.

In our experiment, we ran the candidate pairs just once and did not repeat the experiment to see if we get the same results.
We only ran few malware pair with positive result more than once and got the same result (deletion of resource occurred).
Also, we only ran the candidate pair, \emph{(a,b)}, in single order of execution, such that \emph{`a'} runs first, followed by \emph{`b'}, where \emph{`a'} was supposed to create the resource and \emph{`b'} was supposed to access/delete it.
We do not know if the behavioral interference persists if we run the candidate pair in reverse order; with \emph{`b'} being executed first followed by \emph{`a'}.
We ran some candidate pairs with positive result in reverse order and found that the deletion of resource did not occur if the order of execution was reversed.
However, for both of the above cases, we cannot confirm that our results will always be same, as we did not perform the test in all malware pairs with positive result of behavioral interference; because of the time constraint.

Even though we found the concrete evidence where a malware deletes the resource created by another malware, we still do not know the true semantics of the malware.
% As the results of our experiment, we did find the concrete evidence on deletion of resource, but we still do not know the true semantics of the malware.
The malware could have deleted just just one single resource that could be very common such as \emph{`Desktop.ini`} file.
% Also as shown in~\autoref{sub:Interesting Cases}, a malware could just delete all the files and folder from the drive leading to false positive detection.
Further in depth analysis of individual malware belonging to the candidate pair must be done in order to confirm that the purpose of one malware was indeed to negate the infection of other malware from the system.
\section{Summary}
\label{sec:Summary}

We have shown with our results that our system can find malware pair from different families with behavioral interference.
During the experiment, there were  many aspects that we overlooked and many variables that we controlled to get the current results.
But those threats only challenge the effectiveness of our result, and not the fact that there were malware samples which deletes the resources created by another malware.
With this results, we provide strong support to our hypothesis for the existence of such behavioral interference between the malware of different families, and that our system can detect such behavioral interference.

