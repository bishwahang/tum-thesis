%!TEX root = ../main.tex
\chapter{Literature Review}\label{chapter:literature_review}
To the best of our knowledge there has not been any research on finding the behavioral interaction between the malware families.
There are some anecdotal evidence with some blogs but not any systematic research proving that the change in behavior of a malware on the presence of another do exist.
However, there has been some work on clustering the malware with regard to its behavior. We will discuss in detail about it later.\\
\section{Malware Families}
\label{sec:Malware Families}
We have already discussed about Malware and its types in previous Chapter. Let us look at some of the famous widely spread malware families.\\
\subsection{Conficker}
\label{sub:Conficker}
\textbf{Conficker} is a computer \emph{worm} that targets the \emph{Microsfot Windows} Operating system which was first found in November 2008.
According to \emph{Microsoft}~\cite[Worm:Win32/Conficker.B]{conficker} the detection of \emph{Confiker} worm increased by more that 225 percent since start of 2009.
It is capable of infecting and spreading across network without any human interaction.
It exploits weak password use by trying some common passwords,in order to drop a copy of itself in a network share directory, if current user does not has admin privilege.\\
The worm first tries to copy itself in the Windows system folder as a hidden DLL using some random name. When unsuccessful it tries to copy in \emph{\%ProgramFiles\%} directory.
In order to run on startup, every time the Window boots, it also changes the registry as in~\autoref{lst:confikerregistry}.
\begin{lstlisting}[language=TeX,caption={Registry key created by Confiker worm for autostart},label={lst:confikerregistry}]
In subkey: HKCU\Software\Microsoft\Windows\CurrentVersion\Run
Sets value: "<random string>"
With data: "rundll32.exe <system folder>\<malware file name>.dll,<malware parameters>"
\end{lstlisting}
Not only that it can load itself as service whenever \emph{netsvcs} group is loaded by system \emph{svchost.exe},
it is also capable of loading itself as fake service under \url{`HKLM\\SYSTEM\\CurrentControlSet\\Services'}.
\subsection{Zeus}
\label{sub:Zeus}
\textbf{Zeus} is another malware of \emph{trojan} type, that affects the \emph{Microsoft Windows} Operating System, that attempts to steal confidential information once it infects the victim's machine.
The information may include systems information, banking details, or online credentials of the compromised machine.
It is also capable of downloading configuration files and updates from the Internet. Attacker can used this to change the function of Trojan to change the target value. The Trojan is generated by a toolkit which is also available in underground criminal market, and distribute itself by spam,phising and drive-by downloads.~\cite[Trojan.Zbot]{zeus}.\\
Zeus tries to create a copy of itself as any of the names as \emph{ntos.exe, sdra64.exe,twex.exe} in the \textit{<system folder>}.
In order to make itself run every time the system starts it changes the registry as in~\autoref{lst:zeusregistry}~\cite[Win32/Zbot]{zeusmicro}.
\begin{lstlisting}[language=TeX,caption={Registry key modified by Zeus Trojan to autostart},label={lst:zeusregistry}]
In subkey: HKLM\Software\Microsoft\Windows NT\Currentversion\Winlogon
Sets value: "userinit"
With data: "<system folder>\userinit.exe,<system folder>\<malwar"
\end{lstlisting}
\subsection{Sality}
\label{sub:Sality}
\textbf{Sality} is a family of polymorphic malware that infects files on \emph{Microsfot Windows} system with extensions \emph{.EXE} or \emph{.SCR}.
It spreads by infecting executatble files and replicating itself across network shares.
Each infected host becomes the part of peer to peer network used to propagate the malware~\cite[Sality]{salitysym} \\
\emph{Sality} targets all files in \emph{C:}, and tries to delete files related to anti-virus, stops security related process, and steals sensitive information like cached password and logged keystrokes.
It also changes Windows registry keys in order to lower the computer security.
One of the symptoms to find out if a machine is infected by \emph{Sality} family malware is presence of~\autoref{lst:salityfiles}files in the PC~\cite[Win32/Sality]{salitymicro}.
\begin{lstlisting}[language=TeX,caption={Files created by Sality in the infected machine},label={lst:salityfiles}]
<system folder>\wmdrtc32.dll
<system folder>\wmdrtc32.dl_
\end{lstlisting}
\section{Battle between Malware Families}
\label{sec:Battle between Malware Families}
There has been some evidence of battle between the malware families. We will look upon some of those references in past.
\subsection{Bagle, Netsky, and Mydoom feud}
\label{sub:Bagle, Netsky, and Mydoom feud}
This feud between the malware family dates back to 2004, where there was exchange of words between the creators of \emph{Bagle, Netsky} and \emph{Mydoom}.
The creators inserted their message inside the malware itself, making it visible for the victims too. This also gained much media attention and even appeared in~\cite[BBC]{bbccover}.
Messages like \emph{``dont'r ruine (sic) our business, wanna start a war?''} was seen inside the \emph{Bagle.J}, where as the response message inside \emph{Netsky.F} was \emph{``Bagle \- you are a looser!!!! (sic)''}.
Similar messages with profanity was also seen in variants of \emph{Mydoom.G} families.\\
The reason for the war between these malware families was \emph{Netsky} trying to remove the other two malware \emph{Bagle} and \emph{MyDoom} from the victim's machine.
The creator of \emph{Netsky}, \emph{Sven Jaschan}, also seems to admit that he had written the malware in order to remove infection with \emph{Mydoom} and \emph{Bagle} worms from victim's computer~\cite[]{wikinetsky}.\\
All three malware family were computer worms spreading through email as an attachment and making victim curious enough to download and open it.
\subsection{Kill Zeus}
\label{sub:Kill Zeus}
This was another war between the two \emph{Trojan} malware families \emph{Spy Eye} and \emph{Zeus}.
\emph{SpyEye} is a malware which like \emph{Zeus} was created specifically to facilitate online theft from the financial institutions, especially targeting U.S.
It infected roughly 1.4 million computers, mainly located in U.S, and got the personal identification informations and financial informations of victim in order to transfer money out of victim bank accounts~\cite[]{fbispyeye} \\
\textbf{SpyEye} came with the feature called \emph{Kill Zeus} that was successful in removing large varieties of \emph{Zeus} family.
The battle between these two family was pretty much dynamic and many releases of the Trojan toolkit were made from both the family in order to negate each others affect~\cite[]{sanszeus}.
\subsection{Shifu}
\label{sub:Shifu}
This highly evasive malware family, \textbf{Shifu}, targeting mainly Japanese (82\%), Austria-Germany (12\%), and mix Europe (6\%) banks for sensitive data, was detected recently in the year 2015~\cite[]{secintelshifu}.
It uses \emph{Domain Name Generation} to generated random domain names for covert botnet communication.
It terminates itself it it finds that it is being run inside a virtual machine or is being debugged, which it finds out by checking system files such as \emph{pos.exe, vmmouse.sys, sanboxstarter.exe} or \emph{IsDebuggerPresent} API~\cite[]{mccafeshifu}.\\
\emph{Shifu} had antivirus kind of feature that prevented any other malware from infecting its victim by stopping the installation or making the file look suspicious.
It kept the track of all the files being downloaded by hooking \textit{URLDownloadToFile}\footnote{Downloads bit from Internet.\ \url{https://msdn.microsoft.com/en-us/library/ms775123\%28v=vs.85\%29.aspx}} function.
For any unsigned executables coming from unsecure network it renamed the file to \emph{infected.exx} and also send the copy of file to its \emph{C\&C} server, probably for further analysis.
\emph{Shifu} did not find or delete if the system is prior affected by other malware, but once it had the control of victim's, it managed to stop infection from further malware or any updates to the existing malware by disconnecting it with its botmaster~\cite[]{secintelshifu} 
\section{Malware Analysis}
\label{sec:Malware Analysis}
There are usually two types of malware analysis system, \emph{Static} and \emph{Dynamic}.\\
\textbf{Static Anaylsis} studies the binary by disassembling it with use of disassembler, decompilers, source code analyzers to find the control flow graph of all the code segment and path that the program might take under different given conditions.
The analysis is done without running the program, but studying the assembly code of binary to detect the benign or malicious nature of program.
\emph{Static analysis} gives all the possible behavior of program but for large number of huge programs it is hard and time consuming.
\textbf{Dynamic Analysis} is done by running the binary in a closed controlled system such as \emph{Virtual Box} or \emph{Sandbox} and logging all the activities of the malware while it runs in host.
\emph{Dynamic analysis} is faster than static analysis and gives clear insight of the program behavior and nature.
However it might not be feasible to traverse all the execution path that a program might take.
\\
With the increase in the rate of new malware with polymorphic and metamorphic nature, it has become much difficult to detect a new malware from \emph{static analysis}.
The syntactic signatures are ignorant of the semantics of the instruction and can be easily evaded with code obfuscation and making program control flow obscure~\cite[]{staticlimit}.\\
Also, many environment sensitive malware evade the \emph{dynamic analysis} by changing their behavior, and not performing any malicious activities, once they find out they are being run inside analysis environment.
This makes detection of evasive malware much more manual process~\cite[]{barecloud}.

\section{Malware Clustering}
\label{sec:Malware Clustering}
With the advent of dynamic analysis of malware, there has also been rise in malware clustering research and work.
Many variants of same malware family are on the rise because of polymorphic behavior of malware in order to evade the traditional signature based detection system.
A good clustering system would be able to detect the similarity between malware to find if the new binary is a variant of previous malware and detect the malware.
There are some previous work on clustering the malware based on different parameters.
Different AV company give their own different label to the malware.
A binary which might be labeled as malware may be labeled benign by another AV, or might be labeled as belonging to different family.
There is not consistency between the labeling of malware and also because of this Virus Total labeling is not considered a good parameter.
But, that is only one parameter available for now, which we can use for reference.
