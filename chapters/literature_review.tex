%!TEX root = ../main.tex
\chapter{Literature Review}\label{chapter:literature_review}
In this chapter we will discuss on anecdotal evidences and previous research works and show the necessity of our study to add a new aspect on understanding malware behavior.
We start the chapter with brief summary of some popular malware families in~\autoref{sec:Malware Families} and different incident of interferences between malware families in the past in~\autoref{sec:Interference between Malware Families}.
We discuss about malware analysis techniques and \emph{Anubis} system in~\autoref{sec:Malware Analysis}.
We then discuss why AV vendor are not reliable in case of malware family determination, and how behavioral clustering has been used so far in different research work to cluster similar behaving malware in~\autoref{sec:Behavioral Clustering}.\\
We end the chapter with summary of existing knowledge and motivation for new research.
\section{Malware Families}
\label{sec:Malware Families}
We discussed about Malware and its types in~\autoref{chapter:introduction}.
Malware are categorized to different families in accordance to the malware author and similar behavior.\\
In this section, we briefly describe three most common malware families to provide and idea on their behaviors, capabilities, and mode of operation.
Even between these three families we can see stark difference in their behavior and approach of infecting victim's machine permanently keeping their presence low.
Each malware families created or modified specific resource (such as file or registry) which we can see below.
\subsection{Conficker}
\label{sub:Conficker}
\textbf{Conficker} is a computer \emph{worm} that targets the \emph{Microsfot Windows} Operating system which was first found in November 2008.
According to \emph{Microsoft} the detection of \emph{Conficker} worm increased by more that 225 percent since start of 2009~\cite[]{conficker}.
It is capable of infecting and spreading across network without any human interaction.
If the user of the compromised machine does not has admin privilege, it is capable of trying different common weak passwords (password such as \emph{`test123'} or \emph{`password'}) in order to gain admin privilege of network share directory, and drop a copy of itself.\\
The worm first tries to copy itself in the Windows system folder as a hidden Dynamic Link Library (DLL) using some random name. When unsuccessful it tries to copy itself in \emph{\%ProgramFiles\%} directory.
In order to run on startup, every time the Window boots, it also changes the registry as in~\autoref{lst:confikerregistry}.
\begin{lstlisting}[language=TeX,caption={Registry key created by Confiker worm for autostart},label={lst:confikerregistry}]
In subkey: HKCU\Software\Microsoft\Windows\CurrentVersion\Run
Sets value: "<random string>"
With data: "rundll32.exe <system folder>\<malware file name>.dll,<malware parameters>"
\end{lstlisting}
% Not only that it can load itself as service whenever \emph{netsvcs} group is loaded by system \emph{svchost.exe},
% it is also capable of loading itself as fake service under \url{`HKLM\\SYSTEM\\CurrentControlSet\\Services'}.
\subsection{Zeus}
\label{sub:Zeus}
\textbf{Zeus} is another malware of \emph{Trojan} type, that affects the \emph{Microsoft Windows} OS\@.
It attempts to steal confidential information once it infects the victim's machine.
The information may include systems information, banking details, or online credentials of the compromised machine.
It is also capable of downloading configuration files and updates from the Internet and change its behavior based on new update.
The Trojan is generated by a toolkit which is also available in underground criminal market, and distributes itself by spam,phising and drive-by downloads~\cite[Trojan.Zbot]{zeus}.\\
Zeus tries to create a copy of itself as any of the names as \emph{ntos.exe, sdra64.exe,twex.exe} in the \textit{<system folder>}.
In order to make itself run every time the system starts, after reboot or shutdown, it changes the registry as in~\autoref{lst:zeusregistry}~\cite[Win32/Zbot]{zeusmicro}.
\begin{lstlisting}[language=TeX,caption={Registry key modified by Zeus Trojan to autostart},label={lst:zeusregistry}]
In subkey: HKLM\Software\Microsoft\Windows NT\Currentversion\Winlogon
Sets value: "userinit"
With data: "<system folder>\userinit.exe,<system folder>\<malwar"
\end{lstlisting}
\subsection{Sality}
\label{sub:Sality}
\textbf{Sality} is a family of polymorphic malware that infects files on \emph{Microsfot Windows} OS with extensions \emph{.EXE} or \emph{.SCR}.
It spreads by infecting executable files and replicating itself across network shares and steals sensitive information like cached password and logged keystrokes.A\\
Each infected host becomes the part of peer to peer (p2p) botnet that would create a hard to take down decentralized network of C\&C servers.
The botnet is also used to relay spam, proxy communications, or achieve DDoS~\cite[Sality]{salitysym}\\
\emph{Sality} targets all files in system drive (usually \emph{`C`} drive in Windows) and tries to delete files related to anti-virus.
It stops any security related process (anti-virus) and changes Windows registry keys related to av software in order to lower the computer security.
One of the symptoms to find out if a machine is infected by \emph{Sality} family malware is the presence of files, listed in~\autoref{lst:salityfiles}, in the system~\cite[Win32/Sality]{salitymicro}.
\begin{lstlisting}[language=TeX,caption={Files created by Sality in the infected machine},label={lst:salityfiles}]
<system folder>\wmdrtc32.dll
<system folder>\wmdrtc32.dl_
\end{lstlisting}
\section{Interference between Malware Families}
\label{sec:Interference between Malware Families}
We discussed about malware families and their supposed behavior in previous section.
In this section we show some of the evidences of interference between different malware families, recorded early in year 2004 to latest in year 2015.
We look upon those incidents to know the families involved, nature of interference, and reason behind it.
One of our main hypothesis is finding such behavioral interference between the malware families in large scale.
This anecdotal evidences will serve as a ground truth for our research work.
\subsection{Bagle, Netsky, and Mydoom feud}
\label{sub:Bagle, Netsky, and Mydoom feud}
This feud between the malware family dates back to 2004, where there was exchange of words between the creators of \emph{Bagle, Netsky} and \emph{Mydoom}.
All three malware family were computer worms spreading through email as an attachment and making victim curious enough to download and open it.
The creators inserted their message inside the malware itself, making it visible for the victims too. This also gained much media attention and even appeared in~\cite[BBC]{bbccover}.
Message \emph{``don't ruine (sic) our business, wanna start a war?''} was seen inside the \emph{Bagle.J}, where as \emph{NetSky.F} responded with message \emph{``Bagle \- you are a looser!!!! (sic)''}.
Similar messages with profanity were also seen in variants of \emph{Mydoom.G} families.\\
The reason for the war between these malware families was \emph{Netsky} trying to remove the other two malware \emph{Bagle} and \emph{MyDoom} from the victim's machine.
The creator of \emph{Netsky}, \emph{Sven Jaschan}, admitted that he had written the malware in order to remove infection with \emph{Mydoom} and \emph{Bagle} worms from victim's computer~\cite[]{wikinetsky}.\\
\subsection{Kill Zeus}
\label{sub:Kill Zeus}
This was another war between the two \emph{Trojan} malware families \emph{Spy Eye} and \emph{Zeus}.\\
\textbf{SpyEye} is a Trojan malware which like \emph{Zeus} was created specifically to facilitate online theft from the financial institutions, especially targeting U.S.
It infected roughly 1.4 million computers, mainly located in U.S, and got the personal identification informations and financial informations of victim in order to transfer money out of victim bank accounts~\cite[]{fbispyeye} \\
\textbf{SpyEye} came with the feature called \emph{Kill Zeus} that was successful in removing large varieties of \emph{Zeus} family.
The battle between these two families were pretty much dynamic.
Many releases of the Trojan toolkit were made from both the family in order to negate each others dominance~\cite[]{sanszeus}.
\subsection{Shifu}
\label{sub:Shifu}
\textbf{Shifu}, a highly evasive malware family, targeting mainly Japanese (82\%), Austria-Germany (12\%), and rest of Europe (6\%) banks for sensitive data, was detected recently in the year 2015~\cite[]{secintelshifu}.
It is evasive because it terminates itself if it finds out that it is being run inside the virtual machine or is being debugged.
It knows if it is being run inside a virtual machine by checking the presence of files such as \emph{pos.exe, vmmouse.sys, sanboxstarter.exe} in the system~\cite[]{mccafeshifu}.
In order to check if it is being debugged, it calls \emph{IsDebuggerPresent} Windows API which detects if program is being debugged~\cite[]{mccafeshifu}.\\
\emph{Shifu} also had antivirus kind of feature that prevented any other malware from infecting its victim.
It kept the track of all the files being downloaded from the Internet.
For any files that were downloaded from unsecured connections (not HTTPS) or are not signed, it considered those files suspicious, and renamed it to \emph{infected.exe}.
It stopped those suspicious files from being installed in the system and also sent a copy of the file to its C\&C server, probably for further analysis~\cite[]{secintelshifu}.
If there were already presence of other malware, it would stop any updates to those other malware, by disconnecting them from their botmaster~\cite[]{secintelshifu}.
\section{Malware Analysis}
\label{sec:Malware Analysis}
In previous sections, we talked about malware families and interference between them, in this section we talk about techniques of malware analysis and the system we chose for malware analysis.
%TODO: connect to previous section
\emph{Malware Analysis} is a process of dissecting different components of a malware in order to study the behavior of the malware when it infects a host or victim's machine.
It is done using different analysis tools and reverse engineering the binary.
There are usually two main types of malware analysis techniques, \emph{Static} and \emph{Dynamic}.\\
\textbf{Static Anaylsis} is done without running the binary, but studying the assembly code of binary to detect the benign or malicious nature of program.
One of the way to perform static analysis is by disassembling the binary with use of disassembler, decompilers, source code analyzers to find the control flow graph (CFG) of all the code segment and path that the program might take under different given conditions.
%TODO: other method string based, ngram, entropy based
If the analysis result crosses the preset threshold of malicious activity then it is marked as infected~\cite[]{sharma2014}.
\emph{Static analysis} gives all the possible behavior of program but for large number of huge programs it is hard and time consuming.
\textbf{Dynamic Analysis} is done by running the binary in a closed controlled system such as \emph{Virtual Box} or \emph{Sandbox} and logging all the activities (activity related to Registry, File, Process, Network etc.) of the malware while it runs in host.
We can detect any suspicious activities, such as attempt to open other executables with intent to modify its content,changing Master Boot Record, concealing themselves from the operating system,
% We can look through the log such as attempt to open other executables with intent to modify its content,changing Master Boot Record, concealing themselves from the operating system
% However it might not be feasible to traverse all the execution path that a program might take.
\\
With the increase in the rate of new malware with polymorphic and metamorphic nature, it has become much difficult to detect a new malware from \emph{static analysis}.
The syntactic signatures are ignorant of the semantics of the instruction and can be easily evaded with code obfuscation and making program control flow obscure~\cite[]{staticlimit}.\\
Also, many environment sensitive malware evade the \emph{dynamic analysis} by changing their behavior, and not performing any malicious activities, once they find out they are being run inside analysis environment.
This makes detection of evasive malware much more manual process~\cite[]{barecloud}.

\section{Behavioral Clustering}
\label{sec:Behavioral Clustering}
With the advent of dynamic analysis of malware, there has also been rise in different approach to build a system that could analyze the malware in large numbers.
New malware are on the rise at an exponential rate with millions of them being introduced each day.
Malware authors use polymorphic and metamorphic techniques to make malware more complex and change it's form so as to make traditional signature based detection hard to detect them.
Even dynamic analysis required manual inspection because of diversity in malware variants and its rapid increase.
A good clustering system would be able to detect the similarity between malware to find if the new binary is a variant of previous malware and classify them accordingly.
\subsection{Limitation of Anti Virus System Labels}
\label{sub:Limitation of Anti Virus System Labels}
The need for clustering of malware arises as labels provided by AV companies, such as in \emph{VirusTotal}~\cite[]{virustotal}, are not reliable.
According to~\citeauthor{bailey}, they do not fulfill the criteria consistency, completeness, and conciseness~\cite[]{bailey}.
\begin{itemize}
\item \textbf{Consistency.} Different AV system place the malware into different categories, and these categories also do not hold same meaning across the system.
This leads to inconsistency in the AV system labels.
\item \textbf{Completeness.} Not every malware has been labeled by the AV system, and many malware go undetected because of this.
Many binaries that were months old, were still without any labeled, which makes the AV system labeling incomplete.
\item \textbf{Conciseness.} The label provided by AV system were either too little or too much with not much substantial meaning.
They usually gave general idea of broad term as ``trojan'' and ``worm'' and not concise description to the specific malware or its family.
\end{itemize}
\subsection{Previous Works}
\label{sub:Previous Works}
There have been many approaches to use automated malware clustering in order to classify binaries based on their behavioral similarity.
The result then is used to analyze any new binary to find out if it's novel or similar based on the comparison to previously known malware clusters.\\
Here we discuss about some previous works done on classification of malware based on the behavioral activities of malware.\\
\textbf{\citeauthor{bayer}} used execution traces of malware in order to create a behavioral profile which they later used as an input to their clustering algorithm.
This was the first approach to use clustering of malware on a large scale.
Their system was able to recognize and group similar malware with better precision and scalable manner with the clustering speed of 75 thousands malware in three hours.
\textbf{\citeauthor{autoana}} also used clustering and classification using \emph{machine learning} to analyze behavior of malware.
Their work was inspired by~\cite[]{bayer} and used incremental approach to decrease the runtime overhead, as malware were processed in a daily basis with batch of 1000~\cite[]{autoana}.\\
\textbf{\citeauthor{firma}} developed \emph{FIRMA}, a tool that clusters the malware binary into families, based on the network traffic of unlabeled malware binaries.
It also produces network signature for each of the network behavior of family.
The tool was evaluated against the datasets of \emph{16,000} unique malware binaries~\cite[]{firma}.\\
\textbf{\citeauthor{pirscoveanu}} also presented a system to analyze and classify high amount of malware in a scalable and automatized manner, which could be used pre-filter novel from known malware.
They used~\cite[Cuckoo]{cuckoo} to generate behavioral profile of a malware and \emph{Random Forest} for \emph{supevised} machine learning.
The total number of samples used for experiment were \emph{42,068}, out of which $67\%$ were used for training and remaining $33\%$ were used as testing set~\cite[]{pirscoveanu}.\\
\textbf{\citeauthor{yavvari}} uses behavioral mapping approach to find the commonalities between malware behavior grouped component wise.
They define the term ``soft cluster'' which represents malware relationship with respect to all behavioral similarities, however small.
Using soft clustering approach they revealed component sharing that belonged to different families according to traditional grouping.
The system was evaluated on the set of \emph{1,727} unique malware samples~\cite[]{yavvari}.
\textbf{\citeauthor{firdausi}} in their paper performed different classification using different classifiers, \emph{k-Nearest Neighbor, Naive Bayer, J48 decision trees, Support Vector Machine, and Multilayer Perception Neural Network}, to classify the malware based on behavioral data.
They conclude that machine learning techniques based on behavioral profile could detect malware quite efficiently and effectively~\cite[]{firdausi}.\\
\section{Summary and Motivation}
\label{sec:Motivation}
With the increase of online banking activities, there has been increase in number of new malware, and fighting between them to control the larger share of the underground economy.
In order to do that, they try to negate the existence or influence of another malware once they detect them.
We showed some evidences of interferences between the malware families where they removed other malware, added a feature to remove other malware, or blocked other malware from infecting its victim.\\
Our research will provide a systematic way to find such behavior of malware on a large scale data.
\\
Many research work has been based on clustering of malware based on the behavioral profiles generated by dynamic analysis.
All of them used some machine learning algorithm on the corpus of behavioral profile to cluster the malware, in order to improve the malware detection rate of a system.\\
However, the clustering of malware performed so far are on a small corpus, consisting of tens of thousands of malware samples.
Our clustering is based on much larger samples, {\gettotalmalwarei{}} to be precise.
We present a different approach to malware clustering, for large scale malware samples, which has been discussed in details in~\autoref{sec:Topic model}.\\
We know that different malware family have different techniques to inject themselves and staying alive in the host.
As described in~\autoref{sec:Malware Families}, they copy themselves in certain system path, disguise themselves with benign filenames and register themselves to autostart services.
The names of different resources created by malware may  be random or peculiar.
To select the sample candidate pair from different family, we look for those malware that tried to access or delete certain resource which was created by malware from another family.
We believe such behavior is suspicious and interesting case of interference between two malware family.
% and run them together to detect the battle behavior between them (such as deletion of file created by another malware) and deviation in their behavior (number of files created activities).
The detailed approach of selection of pair is described in~\autoref{sec:Running Experiment}.\\
To our best knowledge, the study of behavioral interference between the malware families has not been done before.
Our work will provide a cornerstone in that topic.
It will help to understand dynamic behavior of environment sensitive malware and working with large malware data samples.
% We believe in the notion that malware belonging to same families will have similar behavioral activities, thus more common resource activity.
% We use unsupervised machine learning algorithm to perform the clustering of those malware on very large data set of millions of malware.
% We will discuss in details about the algorithm we chose for clustering and how we perform it in
% Different AV companies have been trying to label the malware with their own labeling system.
% But, there has not been a common standard for this leading to inconsistency as they use different approach for classification.
% A binary which might be labeled as malware may be labeled benign by another AV, or might be labeled as belonging to different family.
% There is not consistency between the labeling of malware and also because of this Virus Total labeling is not considered a good parameter.
