%!TEX root = ../main.tex
\chapter{Literature Review}\label{chapter:literature_review}
In this chapter we will discuss on anecdotal evidences and previous research works and show the necessity of our study to add a new aspect on understanding malware behavior.
We start the chapter with brief summary of some popular malware families in~\autoref{sec:Malware Families} and different incident of interferences between malware families in the past in~\autoref{sec:Interference between Malware Families}.
We discuss about malware analysis techniques and \emph{Anubis} system in~\autoref{sec:Malware Analysis}.
We then discuss why AV vendor are not reliable in case of malware family determination, and how machine learning techniques has been used so far in different research work to classify and cluster malware in~\autoref{sec:Malware Clustering}.\\
We end the chapter with summary of existing knowledge and motivation for new research.
\section{Malware Families}
\label{sec:Malware Families}
We discussed about Malware and its types in~\autoref{chapter:introduction}.
Malware are categorized to different families in accordance to the malware author and similar behavior.\\
In this section, we briefly describe three most common malware families to provide and idea on their behaviors, capabilities, and mode of operation.
Even between these three families we can see stark difference in their behavior and approach of infecting victim's machine permanently keeping their presence low.
Also each malware families created or modified some specific resource (such as file or registry) which can be used to detect their presence.
\subsection{Conficker}
\label{sub:Conficker}
\textbf{Conficker} is a computer \emph{worm} that targets the \emph{Microsfot Windows} Operating system which was first found in November 2008.
According to \emph{Microsoft} the detection of \emph{Conficker} worm increased by more that 225 percent since start of 2009~\cite[]{conficker}.
It is capable of infecting and spreading across network without any human interaction.
If the user of the compromised machine does not has admin privilege, it is capable of trying different common weak passwords (such as \emph{`test123'},\emph{`password123'}) in order to gain admin privilege of network share directory, and drop a copy of itself.\\
The worm first tries to copy itself in the Windows system folder as a hidden Dynamic Link Library (DLL) using some random name. When unsuccessful it tries to copy itself in \emph{\%ProgramFiles\%} directory.
In order to run on startup, every time the Window boots, it also changes the registry as in~\autoref{lst:confikerregistry}.
\begin{lstlisting}[language=TeX,caption={Registry key created by Confiker worm for autostart},label={lst:confikerregistry}]
In subkey: HKCU\Software\Microsoft\Windows\CurrentVersion\Run
Sets value: "<random string>"
With data: "rundll32.exe <system folder>\<malware file name>.dll,<malware parameters>"
\end{lstlisting}
% Not only that it can load itself as service whenever \emph{netsvcs} group is loaded by system \emph{svchost.exe},
% it is also capable of loading itself as fake service under \url{`HKLM\\SYSTEM\\CurrentControlSet\\Services'}.
\subsection{Zeus}
\label{sub:Zeus}
\textbf{Zeus} is another malware of \emph{Trojan} type, that affects the \emph{Microsoft Windows} OS\@.
It attempts to steal confidential information once it infects the victim's machine.
The information may include systems information, banking details, or online credentials of the compromised machine.
It is also capable of downloading configuration files and updates from the Internet and change its behavior based on new update.
The Trojan is generated by a toolkit which is also available in underground criminal market, and distributes itself by spam,phising and drive-by downloads~\cite[Trojan.Zbot]{zeus}.\\
Zeus tries to create a copy of itself as any of the names as \emph{ntos.exe, sdra64.exe,twex.exe} in the \textit{<system folder>}.
In order to make itself run every time the system starts, after reboot or shutdown, it changes the registry as in~\autoref{lst:zeusregistry}~\cite[Win32/Zbot]{zeusmicro}.
\begin{lstlisting}[language=TeX,caption={Registry key modified by Zeus Trojan to autostart},label={lst:zeusregistry}]
In subkey: HKLM\Software\Microsoft\Windows NT\Currentversion\Winlogon
Sets value: "userinit"
With data: "<system folder>\userinit.exe,<system folder>\<malwar"
\end{lstlisting}
\subsection{Sality}
\label{sub:Sality}
\textbf{Sality} is a family of polymorphic malware that infects files on \emph{Microsfot Windows} OS with extensions \emph{.EXE} or \emph{.SCR}.
It spreads by infecting executable files and replicating itself across network shares and steals sensitive information like cached password and logged keystrokes.A\\
Each infected host becomes the part of peer to peer (p2p) botnet that would create a hard to take down decentralized network of C\&C servers.
The botnet is also used to relay spam, proxy communications, or achieve DDoS~\cite[Sality]{salitysym}\\
\emph{Sality} targets all files in system drive (usually \emph{`C`} drive in Windows) and tries to delete files related to anti-virus.
It stops any security related process (anti-virus) and changes Windows registry keys related to AV software in order to lower the computer security.
One of the symptoms to diagnose if a machine is infected by \emph{Sality} family malware is the presence of files, listed in~\autoref{lst:salityfiles}, in the system~\cite[Win32/Sality]{salitymicro}.
\begin{lstlisting}[language=TeX,caption={Files created by Sality in the infected machine},label={lst:salityfiles}]
<system folder>\wmdrtc32.dll
<system folder>\wmdrtc32.dl_
\end{lstlisting}
\section{Interference between Malware Families}
\label{sec:Interference between Malware Families}
We discussed about malware families and their supposed behavior in previous section.
In this section we show some of the evidences of interference between different malware families, recorded early in 2004 to latest in 2015.
We look upon those incidents to know the families involved, nature of interference, and reason behind it.
One of our main hypothesis is finding such behavioral interference between the malware families in large scale.
This anecdotal evidences will serve as a ground truth for our research work.
\subsection{Bagle, Netsky, and Mydoom feud}
\label{sub:Bagle, Netsky, and Mydoom feud}
This feud between the malware family dates back to 2004, where there was exchange of words between the creators of \emph{Bagle, Netsky} and \emph{Mydoom}.
All three malware family were computer worms spreading through email as an attachment and making victim curious enough to download and open it.
The creators inserted their message inside the malware itself, making it visible for the victims too. This also gained much media attention and even appeared in~\cite[BBC]{bbccover}.
Message \emph{``don't ruine (sic) our business, wanna start a war?''} was seen inside the \emph{Bagle.J}, where as \emph{NetSky.F} responded with message \emph{``Bagle \- you are a looser!!!! (sic)''}.
Similar messages with profanity were also seen in variants of \emph{Mydoom.G} families.\\
The reason for the war between these malware families was \emph{Netsky} trying to remove the other two malware \emph{Bagle} and \emph{MyDoom} from the victim's machine.
The creator of \emph{Netsky}, \emph{Sven Jaschan}, admitted that he had written the malware in order to remove infection with \emph{Mydoom} and \emph{Bagle} worms from victim's computer~\cite[]{wikinetsky}.\\
\subsection{Kill Zeus}
\label{sub:Kill Zeus}
This was another war between the two \emph{Trojan} malware families \emph{Spy Eye} and \emph{Zeus}.\\
\textbf{SpyEye} is a Trojan malware which like \emph{Zeus} was created specifically to facilitate online theft from the financial institutions, especially targeting U.S.
It infected roughly 1.4 million computers, mainly located in U.S, and got the personal identification informations and financial informations of victim in order to transfer money out of victim bank accounts~\cite[]{fbispyeye} \\
\textbf{SpyEye} came with the feature called \emph{Kill Zeus} that was successful in removing large varieties of \emph{Zeus} family.
The battle between these two families were pretty much dynamic.
Many releases of the Trojan toolkit were made from both the family in order to negate each others dominance~\cite[]{sanszeus}.
\subsection{Shifu}
\label{sub:Shifu}
\textbf{Shifu}, a highly evasive malware family, targeting mainly Japanese (82\%), Austria-Germany (12\%), and rest of Europe (6\%) banks for sensitive data, was detected recently in the year 2015~\cite[]{secintelshifu}.
It is evasive because it terminates itself if it finds out that it is being run inside the virtual machine or is being debugged.
It knows if it is being run inside a virtual machine by checking the presence of files such as \emph{pos.exe, vmmouse.sys, sanboxstarter.exe} in the system~\cite[]{mccafeshifu}.
In order to check if it is being debugged, it calls \emph{IsDebuggerPresent} Windows API which detects if program is being debugged~\cite[]{mccafeshifu}.\\
\emph{Shifu} also prevented any other malware from infecting its victim.
It keeps the track of all the files being downloaded from the Internet.
For any files that were downloaded from unsecured connections (not HTTPS) or are not signed, it considered those files suspicious, and renamed it to \emph{infected.exe}.
It stopped those suspicious files from being installed in the system and also sent a copy of the file to its C\&C server, probably for further analysis~\cite[]{secintelshifu}.
If there were already presence of other malware, it would stop any updates to those other malware, by disconnecting them from their botmaster~\cite[]{secintelshifu}.
\section{Malware Analysis}
\label{sec:Malware Analysis}
In previous sections, we talked about malware families and interference between them; in this section we describe techniques of malware analysis and \emph{Anbuis}, the dynamic analysis system, we chose for our research work.\\
%TODO: connect to previous section
\emph{Malware analysis} is a process of dissecting different components of a malware sample in order to study the behavior of the malware when it infects a host or victim's machine.
It is done using different analysis tools and reverse engineering the binary.
There are two main types of malware analysis techniques, \emph{Static} and \emph{Dynamic}.\\
\textbf{Static anaylsis} is done without running the binary, but studying the assembly code of binary to detect the benign or malicious nature of program.
One of the way to perform static analysis is by disassembling the binary with use of disassembler, decompilers, and source code analyzers to find the control flow graph (CFG) of all the code segment and path that the program might take under different given conditions.
%TODO: other method string based, ngram, entropy based
If the analysis result crosses the preset threshold of malicious activity then it is marked as infected~\cite[]{sharma2014}.
\emph{Static analysis} gives all the possible behavior of program but for large number of huge programs it is hard and time consuming.\\
\textbf{Dynamic Analysis} is done by running the binary in a closed controlled system such as \emph{Sandbox} and logging all the activities (activity related to Registry, File, Process, Network etc.) of the malware while it runs in host.
Detection of malicious activities, such as attempt to open other executables with intent to modify its content, changing Master Boot Record or concealing themselves from the operating system, would label the program as malware~\cite[]{sharma2014}.
The main drawback of dynamic analysis is that only single execution path is examined.
\citeauthor{chipounov2012s2e} proposed an approach to multi-path execution by triggering and exploring different execution path during run-time~\cite[]{chipounov2012s2e}.
Also, many evasive malware can evade detection by \emph{dynamic analysis} by identifying the presence of analysis environment, and refraining from performing malicious activities~\cite[]{barecloud}.
% We can look through the log such as attempt to open other executables with intent to modify its content,changing Master Boot Record, concealing themselves from the operating system
% However it might not be feasible to traverse all the execution path that a program might take.
\\
\subsection{Anubis}
\label{sub:Anubis}
For our experiment we use \textbf{Anbuis}, a dynamic malware analysis platform for analyzing Windows PE-executatbles~\cite[]{anubis}.
We needed a dynamic analysis system because we were working on large scale malware dataset to find interference between malware families.
With dynamic analysis system, we could run the possible candidate malware pair and look for the interference between them in real action to give proof to our findings.
It would be tedious work to perform the same task with static analysis for large number of malware.\\
\emph{Anubis} generates a detailed report about modifications made to Windows registry or the file system, about interaction with the Windows Service Manager or other processes and logs of all generated network traffic~\cite[]{anubis}.anubis
We chose \emph{Anbuis} because of the following reasons:
\begin{itemize}
  \item We had direct access to Anubis and multiple instance of Anubis system to perform our malware analysis.
  \item Our dataset of malware were those collected by Anubis over time, and we had access to already generated report of those malware.
  \item It is one of the well accepted dynamic analysis platform among the security researchers.
\end{itemize}
% With the increase in the rate of new malware with polymorphic and metamorphic nature, it has become much difficult to detect a new malware from \emph{static analysis}.
% The syntactic signatures are ignorant of the semantics of the instruction and can be easily evaded with code obfuscation and making program control flow obscure~\cite[]{staticlimit}.\\
% Many malware evade the \emph{dynamic analysis} detection and not performing any malicious activities, once they find out they are being run inside analysis environment.
% This makes detection of evasive malware much more manual process~\cite[]{barecloud}.

\section{Malware Clustering}
\label{sec:Malware Clustering}
In order to find the malware pair that belong to different families and analyze them for behavioral interference in \emph{Anubis}, we need a basis to cluster the malware into different families.
% We believe that the candidate malware pair belonging to two different families will have high probability of interference in terms of negating each other.\\
Anti-virus companies do provide labels to categorize malware into different families (such as listed in report of \emph{VirusTotal}~\cite[]{virustotal}), but they are not reliable.
They have many shortcomings and limitations.
According to~\citeauthor{bailey}, they do not fulfill the criteria consistency, completeness, and conciseness~\cite[]{bailey}.\\
\begin{itemize}
\item \textbf{Consistency.} Different AV vendors place the malware into different categories, and these categories also do not hold same meaning across the vendors.
This leads to inconsistency in the AV vendors labels.
\item \textbf{Completeness.} Not every malware has been labeled by the AV vendors, and many malware go undetected because of this.
A month old binary could still be unlabeled, which makes the AV system labeling incomplete.
\item \textbf{Conciseness.} The label provided by AV vendor are either too little or too much information with not much substantial meaning.
They usually gave general idea of broad term as ``trojan'' and ``worm'' and not concise description of the specific malware or its family.
\end{itemize}
With the advent of dynamic analysis of malware, there has also been rise in different approach to build an automated system that could analyze the malware in large numbers.
New malware are on the rise at an exponential rate with about 400,000 of them being introduced each day in the year 2015~\cite[]{avtest}; and with polymorphic and metamorphic techniques, malware have become complex and hard to detect from traditional signature based system.
% Malware authors use polymorphic and metamorphic techniques to make malware more complex and change its form so as to make traditional signature based detection hard to detect them.
As malware continue to evolve and rise in numbers, researchers are investigating machine learning techniques such as classification and clustering to analyze malware.
\textbf{Classification} is an instance of supervised learning where a trained set of correctly identified data is used as a reference to classify new data.
\textbf{Clustering}, on other hand is an instance of unsupervised learning, i.e., grouping unlabeled data object which are similar in some sense in same cluster.
A good clustering system would be able to study new binary to find if the new binary is a variant of previous known malware and classify them accordingly.\\
As we described earlier that AV label on malware are not much reliable, we look onto different clustering techniques to categorize our dataset into different probable families.
Let us discuss about works done so far on clustering and classification of malware to know about current state of art.\\
% We looked on many previous works that were based on clustering of malware (\cite[]{mosko},\cite[]{bayer},\cite[]{firma},\cite[]{pirscoveanu},\cite[]{yavvari}).
% We looked upon some of the previous workAll of the paper had different approach and number of dataset for their work.
\textbf{\citeauthor{pirscoveanu}} used \emph{Random Forest} classifier on the behavior data set of \emph{42,068} malware to achieve high classification rate with weighted average Area Under Curve (AUC) value of 0.98~\cite[]{pirscoveanu}.
\textbf{\citeauthor{mosko}} used \emph{Naive Bayes} classifier based on 20 malware-behavior to detect malware with mean detection accuracy of over 90\%~\cite[]{mosko}.
\textbf{\citeauthor{yavvari}} used behavioral mapping approach to cluster \emph{1,727} unique malware pair samples~\cite[]{yavvari}.
\textbf{\citeauthor{bayer}} used behavior clustering to cluster 75 thousand sample claiming to achieve better precision than the previous approaches~\cite[]{bayer}.
\textbf{\citeauthor{firdausi}} in their paper did the measurement of five different classifiers, \emph{k-Nearest Neighbor, Naive Bayer, J48 decision trees, Support Vector Machine, and Multilayer Perception Neural Network}, to classify the malware based on behavioral data.
With their result, they state that machine learning techniques based on behavioral profile could detect malware quite efficiently and effectively~\cite[]{firdausi}.\\
Based on the previous research work and its result, we believe that machine learning techniques are good approach for malware clustering to cluster malware into different families.
We studied the work of~\citeauthor{bayer} in detail to see if we could use their approach of behavioral clustering as their work were the only one to cluster malware in large scale, i.e. 75 thousands malware under three hours~\cite[]{bayer}.
% Also, they too used \emph{Anubis} system as dynamic analysis platform for their work to create the behavioral profile and we had access to those behavioral profiles.\\
% To cluster the malware we had multiple options. We looked into the behavioral-clustering paper by~\citeauthor{bayer}.
With further study of the~\cite[Bayer]{bayer} paper, we found that their approach still was scalable to millions of samples we needed to cluster.
% The clustering results were for tens of thousands of malware samples and not millions, and the approach still does not look like scalable to millions of samples.
The approach has a linear bootstrapping phase of Locality-sensitive hashing (LSH), after which the $O(n^2)$ hierarchical clustering starts.
The whole premise of faster execution lies within careful tuning of multiple parameters and hash functions (to make the initial phase take care of most of the load), which had not been done by the authors for millions of samples (the biggest execution they had was for 75,000 samples).
This means that we would have needed to tune and change the code as we go forward to get the results we hope for.
Instead we preferred using something ready to use (not to spend time improving something that would not be considered any novelty for our work).
\\
% Another alternative was VirusTotal labels, but unfortunately they are not very reliable as we discussed early.
\\
% A third approach was to use a clustering algorithm that is scalable, but maybe less accurate than the behavioral clustering, but able to cluster millions of samples.
Another approach was to use a clustering algorithm that is scalable and capable to cluster millions of samples.
We decided to map the problem to document clustering, considering each malware as a document, and their behavioral profile as the words in document.
As \textbf{term frequency-inverse document frequency} (\emph{tf-idf} is statistically measured weight to evaluate the importance of a word in document with respect to whole corpus) approaches have a large memory footprint, $O(\#docs \times \#words)$, we switched to clustering algorithms whose memory footprint does not depend on the number of documents, and decided to use \textit{latent Dirichlet allocation} (\textbf{LDA}).
\\
\subsection{Latend Dirichlet Allocation}
\label{sub:Latend Dirichlet Allocation}
\textit{latent Dirichlet allocation} (\textbf{LDA}), is a generative probabilistic model for collections of discrete data such as text corpora developed by~\citeauthor{Blei}~\cite[LDA]{Blei}.
\textbf{LDA}, equivalent to dimension-reduction algorithms for high-dimensional clustering, is one such algorithm which does not depend on number of documents and its memory footprint is $O(\#words\times \#clusters)$.
This gives us a more fine grained clustering compared to previously proposed LHS (Locality-sensitive hashing) based approach.\\
% We will show the implementation and result of LDA, using python library \emph{Gensim} for clustering malware in~\autoref{sec:Document Clustering}.\\
\subsection{Gensim}
\label{sub:Gensim}
\emph{``Gensim''} (\textbf{generate similar})~\cite[]{gensim} is an efficient python library developed by~\citeauthor{gensim}.
We used \emph{``Gensim''}to perform the LDA for clustering malware into families.
We preferred using it because of its simplicity, well documented API, and ability to work on large corpus.\\
We use the \emph{ldamulticore}\cite[]{ldamulticore}  models of the Gensim API.\\
Some of the benefits of using the \emph{ldamulticore} model were, the model utilized the multi cores processor of the machine efficiently with parallelization making the clustering process faster.
The training algorithm is streamed and we could feed the input documents sequentially even for large data.
The training algorithm runs in constant memory with respect to number of documents.
This made possible for us to process corpora that was larger than the memory size of our machine (32 Gigabytes), as size of training corpus did not affect the memory footprint.
The maximum size of our corpus was 81 Gigabytes.

\section{Summary and Motivation}
\label{sec:Motivation}
Motivated by the economic profit, there has been increase in number of new malware, and fighting between the malware families to control the larger share of the underground economy.
To get the larger pie of the economy and show their superiority, malware family try to negate the existence and influence of another family.
We showed some evidences of interferences between the malware families where they removed other malware, added a feature to remove other malware, or blocked other malware from infecting its victim.
Such behavioral interference between the malware families is an interesting case to study and would provide a novel aspect of dynamic behavior of evolving malware.
Our research provides a systematic way to find such behavioral interference of malware families on a large scale data.
\\
We gave an overview on two common malware analysis system and reason for choosing \emph{Anbuis}, a dynamic malware analysis platform, as we could run our candidate pairs in real time to check for behavioral interference.\\
We see that different malware family have different techniques to inject themselves and keep themselves alive even after the system reboot.
As described in~\autoref{sec:Malware Families}, they copy themselves in certain system path, disguise themselves with benign filenames and register themselves to autostart services.
The names of different resources created by malware can be random or peculiar.
These pattern related to each families could be used to check the presence of particular family.
We use this rationale in our heuristics to filter probable malware candidate with behavioral interference.
To select the sample candidate pair from different family, we look for those malware that tried to access or delete certain resource which was created by malware from another family.
The detailed approach of selection of pair is described in~\autoref{sec:Running Experiment}.\\
We believe such behavior of malware is suspicious and interesting case of interference between two malware family.\\
We discussed why AV Vendors labeling of malware family are unreliable and how we will be using machine learning to achieve the clustering of malware to families.
We discussed about use of machine learning techniques in the previous research work for clustering and classification of malware based on their behavioral profiles.
The results of the work so for on classification and clustering of malware were pretty good in order to improve efficiency and detection of the malware analysis system.
However, the clustering of malware performed so far are on a small number of sample, consisting of tens of thousands of malware samples, and were not scalable.
We present a different approach for large scale malware clustering, using LDA for document clustering and python library \emph{Gemsim} for the implementation.\\
Our working dataset is of size {\gettotalmalwarei{}} malware samples.
% and run them together to detect the battle behavior between them (such as deletion of file created by another malware) and deviation in their behavior (number of files created activities).
The study of behavioral interference between the malware families in this large scale is a new research work in the field of malware study.
Our work will be a cornerstone in that topic.
It will help to understand dynamic behavior of environment sensitive malware.
